Zarówno sam język graficzny {\bf Salvador}, jak i jego {\bf interpreter} zostały stworzone przez {\bf Tomasza Ducina}. Poniższy tekst pochodzi ze wstępu do rozdziału 2, {\em \char`\"{}Język Salvador\char`\"{}\/}, pracy magisterskiej Tomasza Ducina pt. {\em \char`\"{}Języki ezoteryczne Piet i Salvador jako uniwersalne maszyny obliczeniowe\char`\"{}\/}.

Salvador jest językiem programowania, który zawdzięcza swą nazwę najgenialniejszemu surrealistycznemu malarzowi wszech czasów, Katalończykowi Salvadorowi Dalemu. Potrafił on przekazać swoje bardzo kontrowersyjne, odważne i konkretne idee bez względu na temat i realizację poszczególnych obrazów. Podobnie jest z językiem Salvador – dosłownie każdy obraz można zaadaptować w taki sposób, by wykonywał z góry określone programy.

Wspólnych cech z Pietem jest niewiele, np. Salvador posiada głowicę (a nawet dwie) czytające obrazy graficzne. Podkreślić jedna należy różnice koncepcyjne względem Pieta. Założeniem przy powstawaniu języka było maksymalne zbliżenie maszyny go interpretującej do maszyny Turinga – rezygnacja ze stosu i zastąpienie jej kolejnym obrazem. Zrezygnowano też z operacji wejścia/wyjścia, przeróżne operacje arytmetyczno-logiczne na danych zastąpiono podstawowymi instrukcjami zerowania, następnika i poprzednika danej wartości – wszystko doprowadzono do możliwie najprostszych operacji.

Instrukcje w Salvadorze są wyznaczane na podstawie konkretnego piksla: ilekroć głowica wskaże tenże piksel, zawsze ta sama instrukcja zostanie wykonana – nie istnieją żadne zależności od otoczenia.

Pisanie programów w Salvadorze odbywa się na zupełnie innej zasadzie niż w Piecie. Można z łatwością nie tylko pisać różne fragmenty kodu i podprogramy niezależnie – i potem je łączyć w gotowe programy, ale również nie trzeba przykładać tak wielkiej wagi do kształtu obrazu. Nie potrzeba patrzeć na kod całościowo przy pisaniu pojedynczej instrukcji (co utrudniało pracę z Pietem). Cały kod jest organizowany w tzw. „siatkę kodu”, byt niezależny od wszelkich obrazów (w Piecie zaś nie sposób używać jakiejkolwiek notacji/symboliki do zapisywania pojedynczych instrukcji, od początku trzeba operować na całej planszy piksli). 